% Options for packages loaded elsewhere
\PassOptionsToPackage{unicode}{hyperref}
\PassOptionsToPackage{hyphens}{url}
%
\documentclass[
]{article}
\usepackage{amsmath,amssymb}
\usepackage{lmodern}
\usepackage{iftex}
\ifPDFTeX
  \usepackage[T1]{fontenc}
  \usepackage[utf8]{inputenc}
  \usepackage{textcomp} % provide euro and other symbols
\else % if luatex or xetex
  \usepackage{unicode-math}
  \defaultfontfeatures{Scale=MatchLowercase}
  \defaultfontfeatures[\rmfamily]{Ligatures=TeX,Scale=1}
\fi
% Use upquote if available, for straight quotes in verbatim environments
\IfFileExists{upquote.sty}{\usepackage{upquote}}{}
\IfFileExists{microtype.sty}{% use microtype if available
  \usepackage[]{microtype}
  \UseMicrotypeSet[protrusion]{basicmath} % disable protrusion for tt fonts
}{}
\makeatletter
\@ifundefined{KOMAClassName}{% if non-KOMA class
  \IfFileExists{parskip.sty}{%
    \usepackage{parskip}
  }{% else
    \setlength{\parindent}{0pt}
    \setlength{\parskip}{6pt plus 2pt minus 1pt}}
}{% if KOMA class
  \KOMAoptions{parskip=half}}
\makeatother
\usepackage{xcolor}
\usepackage[margin=1in]{geometry}
\usepackage{color}
\usepackage{fancyvrb}
\newcommand{\VerbBar}{|}
\newcommand{\VERB}{\Verb[commandchars=\\\{\}]}
\DefineVerbatimEnvironment{Highlighting}{Verbatim}{commandchars=\\\{\}}
% Add ',fontsize=\small' for more characters per line
\usepackage{framed}
\definecolor{shadecolor}{RGB}{248,248,248}
\newenvironment{Shaded}{\begin{snugshade}}{\end{snugshade}}
\newcommand{\AlertTok}[1]{\textcolor[rgb]{0.94,0.16,0.16}{#1}}
\newcommand{\AnnotationTok}[1]{\textcolor[rgb]{0.56,0.35,0.01}{\textbf{\textit{#1}}}}
\newcommand{\AttributeTok}[1]{\textcolor[rgb]{0.77,0.63,0.00}{#1}}
\newcommand{\BaseNTok}[1]{\textcolor[rgb]{0.00,0.00,0.81}{#1}}
\newcommand{\BuiltInTok}[1]{#1}
\newcommand{\CharTok}[1]{\textcolor[rgb]{0.31,0.60,0.02}{#1}}
\newcommand{\CommentTok}[1]{\textcolor[rgb]{0.56,0.35,0.01}{\textit{#1}}}
\newcommand{\CommentVarTok}[1]{\textcolor[rgb]{0.56,0.35,0.01}{\textbf{\textit{#1}}}}
\newcommand{\ConstantTok}[1]{\textcolor[rgb]{0.00,0.00,0.00}{#1}}
\newcommand{\ControlFlowTok}[1]{\textcolor[rgb]{0.13,0.29,0.53}{\textbf{#1}}}
\newcommand{\DataTypeTok}[1]{\textcolor[rgb]{0.13,0.29,0.53}{#1}}
\newcommand{\DecValTok}[1]{\textcolor[rgb]{0.00,0.00,0.81}{#1}}
\newcommand{\DocumentationTok}[1]{\textcolor[rgb]{0.56,0.35,0.01}{\textbf{\textit{#1}}}}
\newcommand{\ErrorTok}[1]{\textcolor[rgb]{0.64,0.00,0.00}{\textbf{#1}}}
\newcommand{\ExtensionTok}[1]{#1}
\newcommand{\FloatTok}[1]{\textcolor[rgb]{0.00,0.00,0.81}{#1}}
\newcommand{\FunctionTok}[1]{\textcolor[rgb]{0.00,0.00,0.00}{#1}}
\newcommand{\ImportTok}[1]{#1}
\newcommand{\InformationTok}[1]{\textcolor[rgb]{0.56,0.35,0.01}{\textbf{\textit{#1}}}}
\newcommand{\KeywordTok}[1]{\textcolor[rgb]{0.13,0.29,0.53}{\textbf{#1}}}
\newcommand{\NormalTok}[1]{#1}
\newcommand{\OperatorTok}[1]{\textcolor[rgb]{0.81,0.36,0.00}{\textbf{#1}}}
\newcommand{\OtherTok}[1]{\textcolor[rgb]{0.56,0.35,0.01}{#1}}
\newcommand{\PreprocessorTok}[1]{\textcolor[rgb]{0.56,0.35,0.01}{\textit{#1}}}
\newcommand{\RegionMarkerTok}[1]{#1}
\newcommand{\SpecialCharTok}[1]{\textcolor[rgb]{0.00,0.00,0.00}{#1}}
\newcommand{\SpecialStringTok}[1]{\textcolor[rgb]{0.31,0.60,0.02}{#1}}
\newcommand{\StringTok}[1]{\textcolor[rgb]{0.31,0.60,0.02}{#1}}
\newcommand{\VariableTok}[1]{\textcolor[rgb]{0.00,0.00,0.00}{#1}}
\newcommand{\VerbatimStringTok}[1]{\textcolor[rgb]{0.31,0.60,0.02}{#1}}
\newcommand{\WarningTok}[1]{\textcolor[rgb]{0.56,0.35,0.01}{\textbf{\textit{#1}}}}
\usepackage{graphicx}
\makeatletter
\def\maxwidth{\ifdim\Gin@nat@width>\linewidth\linewidth\else\Gin@nat@width\fi}
\def\maxheight{\ifdim\Gin@nat@height>\textheight\textheight\else\Gin@nat@height\fi}
\makeatother
% Scale images if necessary, so that they will not overflow the page
% margins by default, and it is still possible to overwrite the defaults
% using explicit options in \includegraphics[width, height, ...]{}
\setkeys{Gin}{width=\maxwidth,height=\maxheight,keepaspectratio}
% Set default figure placement to htbp
\makeatletter
\def\fps@figure{htbp}
\makeatother
\setlength{\emergencystretch}{3em} % prevent overfull lines
\providecommand{\tightlist}{%
  \setlength{\itemsep}{0pt}\setlength{\parskip}{0pt}}
\setcounter{secnumdepth}{-\maxdimen} % remove section numbering
\ifLuaTeX
  \usepackage{selnolig}  % disable illegal ligatures
\fi
\IfFileExists{bookmark.sty}{\usepackage{bookmark}}{\usepackage{hyperref}}
\IfFileExists{xurl.sty}{\usepackage{xurl}}{} % add URL line breaks if available
\urlstyle{same} % disable monospaced font for URLs
\hypersetup{
  pdftitle={While},
  pdfauthor={Guillermo Durán González},
  hidelinks,
  pdfcreator={LaTeX via pandoc}}

\title{While}
\author{Guillermo Durán González}
\date{2022-10-04}

\begin{document}
\maketitle

\hypertarget{aplicaciones-de-while}{%
\subsection{Aplicaciones de While}\label{aplicaciones-de-while}}

El método de Newton es un método numérico popular para encontrar una
raíz de una ecuación algebraica

Nos referimos, a encontrar un cero de la función \(f(x)\)

\[f(x)=0\] Si \(f(x)\) tiene derivada \(f′(x)\), entonces la siguiente
iteración debería converger a una raíz si se empieza lo suficientemente
cerca a esta.

\[x_0 = valor-inicial\]

\[
x_n=x_{n-1}-\frac{f\left(x_{n-1}\right)}{f^{\prime}\left(x_{n-1}\right)} .
\] Esto se basa en la aproximación de Taylor

\[
f\left(x_n\right) \approx f\left(x_{n-1}\right)+\left(x_n-x_{n-1}\right) f^{\prime}\left(x_{n-1}\right) .
\] El método de Newton equivale a encontrar

\[f\left(x_n\right)=0\]

para \(x_n\).

El peincipio es aproximarnos a ese \(x_n\) el más cercano a la raíz.

Consideremos para este caso una función polinomica de grado 3:

\[
f(x)=x^3+2 x^2-7
\]

Donde:

\[
x_n=x_{n-1}-\frac{x_{n-1}^3+2 x_{n-1}^2-7}{3 x^2+4 x}
\] \#\# Su implementación en R es algo así:

\(x\) \textless- \(x_0\)

\(f\) \textless- \(xˆ3+2*xˆ2-7\)

\(tolerancia\) \textless- \(0.000001\)

\(while (abs(f) > tolerancia )\{\)

\(f.prime\) \textless- \(3*xˆ2+4*x\)

\(x\) \textless- \(x-f/f.prime\)

\(f\) \textless- \(xˆ3+2*xˆ2-7\}\)

\(x\)

Para \$x\_0 = 1.5 \$

Tendriamos lo siguiente:

\begin{Shaded}
\begin{Highlighting}[]
\NormalTok{f}\OtherTok{\textless{}{-}}\FunctionTok{double}\NormalTok{(}\DecValTok{1}\NormalTok{)}
\NormalTok{f.derivada}\OtherTok{\textless{}{-}}\FunctionTok{double}\NormalTok{(}\DecValTok{1}\NormalTok{)}
\NormalTok{x}\OtherTok{\textless{}{-}}\FunctionTok{double}\NormalTok{(}\DecValTok{1}\NormalTok{)}
\NormalTok{tolerancia}\OtherTok{\textless{}{-}}\FunctionTok{double}\NormalTok{(}\DecValTok{1}\NormalTok{)}
\NormalTok{x}\OtherTok{\textless{}{-}}\FloatTok{1.5}
\NormalTok{f }\OtherTok{\textless{}{-}}\NormalTok{ x}\SpecialCharTok{\^{}}\DecValTok{3} \SpecialCharTok{+} \DecValTok{2}\SpecialCharTok{*}\NormalTok{x}\SpecialCharTok{\^{}}\DecValTok{2} \SpecialCharTok{{-}} \DecValTok{7}
\NormalTok{tolerancia}\OtherTok{\textless{}{-}} \FloatTok{0.000001}
\ControlFlowTok{while}\NormalTok{(}\FunctionTok{abs}\NormalTok{(f)}\SpecialCharTok{\textgreater{}}\NormalTok{ tolerancia) }
\NormalTok{\{}
\NormalTok{  f.derivada }\OtherTok{\textless{}{-}} \DecValTok{3}\SpecialCharTok{*}\NormalTok{x}\SpecialCharTok{\^{}}\DecValTok{2}\SpecialCharTok{+}\DecValTok{4}\SpecialCharTok{*}\NormalTok{x}
\NormalTok{  x }\OtherTok{\textless{}{-}}\NormalTok{ x}\SpecialCharTok{{-}}\NormalTok{f}\SpecialCharTok{/}\NormalTok{f.derivada}
\NormalTok{  f }\OtherTok{\textless{}{-}}\NormalTok{ x}\SpecialCharTok{\^{}}\DecValTok{3}\SpecialCharTok{+}\DecValTok{2}\SpecialCharTok{*}\NormalTok{x}\SpecialCharTok{\^{}}\DecValTok{2{-}7}
\NormalTok{  \}}
\NormalTok{x}
\end{Highlighting}
\end{Shaded}

\begin{verbatim}
## [1] 1.428818
\end{verbatim}

En consecuencia:

\[
f(1.428818)=1.428818^3+2 \cdot 1.428818^2-7
\] Resulta:

\begin{Shaded}
\begin{Highlighting}[]
\NormalTok{f }\OtherTok{\textless{}{-}} \FloatTok{1.428818}\SpecialCharTok{\^{}}\DecValTok{3} \SpecialCharTok{+} \DecValTok{2}\SpecialCharTok{*}\NormalTok{(}\FloatTok{1.428818}\NormalTok{)}\SpecialCharTok{\^{}}\DecValTok{2{-}7}
\NormalTok{f}
\end{Highlighting}
\end{Shaded}

\begin{verbatim}
## [1] 3.530859e-06
\end{verbatim}

Es decir \(f(1.428818)\) es \(\approx 0.000003530859\)

Ahora podemos ver el mismo ejemplo con repeat:

\begin{Shaded}
\begin{Highlighting}[]
\NormalTok{f}\OtherTok{\textless{}{-}}\FunctionTok{double}\NormalTok{(}\DecValTok{1}\NormalTok{)}
\NormalTok{f.derivada}\OtherTok{\textless{}{-}}\FunctionTok{double}\NormalTok{(}\DecValTok{1}\NormalTok{)}
\NormalTok{x}\OtherTok{\textless{}{-}}\FunctionTok{double}\NormalTok{(}\DecValTok{1}\NormalTok{)}
\NormalTok{tolerancia}\OtherTok{\textless{}{-}}\FunctionTok{double}\NormalTok{(}\DecValTok{1}\NormalTok{)}
\NormalTok{x }\OtherTok{\textless{}{-}} \FloatTok{1.5}
\NormalTok{tolerancia}\OtherTok{\textless{}{-}} \FloatTok{0.000001}
\ControlFlowTok{repeat}\NormalTok{\{}
\NormalTok{  f }\OtherTok{\textless{}{-}}\NormalTok{x}\SpecialCharTok{\^{}}\DecValTok{3} \SpecialCharTok{+} \DecValTok{2}\SpecialCharTok{*}\NormalTok{x}\SpecialCharTok{\^{}}\DecValTok{2} \SpecialCharTok{{-}} \DecValTok{7}
  \ControlFlowTok{if}\NormalTok{( }\FunctionTok{abs}\NormalTok{(f)}\SpecialCharTok{\textless{}}\NormalTok{ tolerancia ) }\ControlFlowTok{break} \CommentTok{\# para quebrar el loop}
\NormalTok{  f.derivada}\OtherTok{\textless{}{-}} \DecValTok{3} \SpecialCharTok{*}\NormalTok{ x}\SpecialCharTok{\^{}}\DecValTok{2}   \SpecialCharTok{+} \DecValTok{4} \SpecialCharTok{*}\NormalTok{ x}
\NormalTok{  x }\OtherTok{\textless{}{-}}\NormalTok{ x }\SpecialCharTok{{-}}\NormalTok{ f }\SpecialCharTok{/}\NormalTok{ f.derivada}
\NormalTok{  \}}
\NormalTok{x}
\end{Highlighting}
\end{Shaded}

\begin{verbatim}
## [1] 1.428818
\end{verbatim}

\end{document}
